\PassOptionsToPackage{dvipsnames, table}{xcolor}
\documentclass[dvipsnames, table, 11pt]{article}

\textwidth=7in
\textheight=9.5in
\hoffset=-1in
\voffset=-1in
\parskip=10pt
\parindent=0pt

\usepackage{style}

\begin{document}
     \title{A numerical estimate of escape rates}

     \author[*]{\textbf{PhD student:} Davide Papapicco}
     \author[*]{\authorcr \textbf{Supervision:} Graham Donovan}
     \author[*]{Lauren Smith}
     \author[$\dagger$]{Merryn Tawhai}

     \affil[*]{Department of Mathematics, Faculty of Science, University of Auckland, Auckland, New Zealand}
     \affil[$\dagger$]{Auckland Bioengineering Institute, University of Auckland, Auckland, New Zealand}
 
     \date{}

     \maketitle

     \begin{abstract}
             Given an overdamped particle in a metastable scalar potential, whose evolution obeys Langevin dynamics, there will be a non-zero probability of it to cross critical thresholds and visits multiple basins of stabilities.
             These escapes are, in the low-noise limit, rare and quantified by Kramer's approximation. 
             A novel adaptation of this setup is recently being involved in the theory of tipping points in dynamical systems. 
             We will present a numerical method that can approximate these tipping points based on timeseries data alone by leveraging the relationship between the escape rate and the scalar potential.
     \end{abstract}
\end{document}
