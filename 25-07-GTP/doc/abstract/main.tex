\PassOptionsToPackage{dvipsnames, table}{xcolor}
\documentclass[dvipsnames, table, 11pt]{article}

\textwidth=7in
\textheight=9.5in
\hoffset=-1in
\voffset=-1in
\parskip=10pt
\parindent=0pt

\usepackage{style}

\begin{document}
     \title{A probabilistic early-warning signal for noise-induced tipping in quasi-stationary regimes of metastability}

     \author[*]{\textbf{PhD student:} Davide Papapicco}
     \author[*]{\authorcr \textbf{Supervision:} Graham Donovan}
     \author[*]{Lauren Smith}
     \author[$\dagger$]{Merryn Tawhai}

     \affil[*]{Department of Mathematics, Faculty of Science, University of Auckland, Auckland, New Zealand}
     \affil[$\dagger$]{Auckland Bioengineering Institute, University of Auckland, Auckland, New Zealand}
 
     \date{}

     \maketitle

     \begin{abstract}
             A lot has been done, in the past $30$ years, in the mathematical characterisation of tipping points and the discovery of early-warning signals (EWS).
             Model specific signals have been derived for $1-$dimensional stochastic dynamical systems while recent endevours are trying to generalise such analytical framework in higher dimensions and even in spatially-extended systems.
             However some unresolved questions still plague the full application of EWS to real data, where information is often scarse and irregular.
             Issues concerning the interpretability, optimality and universaility of EWS for a given timeseries are deeply rooted in the current limitations of analytical methods of complex dynamical systems.
             In particular, in scientific scenarios where information about the model generating the timeseries is limited and missed alarms are critical, there is still a degree of uncertainty to what EWS has to be used as well as how to interpret the trends in the signal itself.
             In this talk we use the missed alarm given by monitoring the temporal variance of the spatial mean of a lattice-based simulation to motivate the ongoing development of a probabilistic EWS for tipping points driven by noise.
             Such EWS is based on Kramer's escape formula from statistical mechanics and it relies on timeseries data alone.
             The derivation relies on a number of assumptions which include the existence of alternative stable states in saddle-node form and, importantly, the quasi-stationarity of the bifurcation parameter.
             Statistical techniques such as detrending via empirical mode decomposition (EMD) and inference of the potential landscape via linear least-squares (LLS) allows us to build a numerical method whose pipeline, we hope, can be readily and easily used by scientits and policy-makers when monitoring timeseries.
             Optimistically, error bounds in the approximations show fast convergence with the number of data points available in the timeseries, which is of extremely practical importance for predicting noise-induced tipping points from real-world data.
             Preliminary results regarding applications of the method to higher-dimensional dynamical systems are also discussed via numerical simulations.
     \end{abstract}
\end{document}
