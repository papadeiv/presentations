\documentclass[../main.tex]{subfiles}
\pagenumbering{gobble}
\begin{document}

\title{Slowly, then all at once: Uncovering the dynamics of a catastrophe}

\author{Davide Papapicco}
\affil{Department of Mathematics, University of Auckland, New Zealand}

\date{}

\maketitle

\begin{abstract}
        Many natural and human complex systems evolve on a slow timescale and are stable with respect to external perturbations.
        However, these systems can experience sudden rapid departures from their natural equilibrium, known as tipping events, which often bring catastrophic, unrecoverable repercussions.
        Extreme paleoclimate events, ecosystems' collapse and economic crises are some examples of dynamical systems evolving slowly around an equilibrium until a tipping point causes a fast critical transition outside the basin of attraction and onto a new, unhealthy state.
        Given the disruption of natural equilibria and the potential unrecoverability of certain states past the critical transitions, forewarning of these tipping points has been the subject of extensive research for the past $30$ years.
        Characterisation of these events and their early-warning signals starts with a dynamical interpretation of these different regimes and further develops into the realm of stochastic processes and transitional states.
        Numerous precursors have been hypothesized and statistical measures have been derived as leading indicators of tipping events for simplified and low-dimensional dynamical systems. 
        Despite these efforts several fundamental issues still plague the practical application of early-warning signals in natural timeseries, with their lack of consistency across a broad spectrum of real-world tipping points posing a major shortcoming in their reliability.
        The purpose of this talk is twofold. In the first part we will address the challenges facing the generalisation of frameworks of tipping events to spatially-extended (high-dimensional) models.
        Subsequentially we will show how a novel, prototypical approach based on the finite-states probabilistic interpretation of critical transitions can potentially address the fallacies of previously proposed, model-based indicators.
\end{abstract}

\end{document}
