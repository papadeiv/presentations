\PassOptionsToPackage{dvipsnames, table}{xcolor}
\documentclass[dvipsnames, table, 11pt]{article}

\textwidth=7in
\textheight=9.5in
\hoffset=-1in
\voffset=-1in
\parskip=10pt
\parindent=0pt

\usepackage{style}

\begin{document}
     \title{Hints of collapse: early-warnings of impending catastrophes}

     \author[*]{\textbf{PhD student:} Davide Papapicco}
     \author[*]{\authorcr \textbf{Supervision:} Graham Donovan}
     \author[*]{Lauren Smith}
     \author[$\dagger$]{Merryn Tawhai}

     \affil[*]{Department of Mathematics, Faculty of Science, University of Auckland, Auckland, New Zealand}
     \affil[$\dagger$]{Auckland Bioengineering Institute, University of Auckland, Auckland, New Zealand}
 
     \date{}

     \maketitle

     \begin{abstract}
             In the second part of our journey through catastrophic collapses in complex systems we will address the potential existence and detection of early-warning signals (EWS).
             Albeit lacking a rigorous definition, EWS are usually interpreted as ``\textit{simple properties that change in characteristic ways prior to a critical transitions}''.
             A mathematical framework to derive and identify EWS has only been formalised in 2011 and since then some progress has been made.
             We will thus review the latest developments in the literature as well as establishing mathematical justification for EWS to exist and be detected from raw timeseries data.
             In conclusion we will also briefly understand how some tools from statistical physics can be used to infer the probability of tipping events in simplified stochastic models.
     \end{abstract}
\end{document}
