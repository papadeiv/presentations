\PassOptionsToPackage{dvipsnames, table}{xcolor}
\documentclass[dvipsnames, table, 11pt]{article}

\textwidth=7in
\textheight=9.5in
\hoffset=-1in
\voffset=-1in
\parskip=10pt
\parindent=0pt

\usepackage{style}

\begin{document}
     \title{Hints of collapse: the theory of tipping points}

     \author[*]{\textbf{PhD student:} Davide Papapicco}
     \author[*]{\authorcr \textbf{Supervision:} Graham Donovan}
     \author[*]{Lauren Smith}
     \author[$\dagger$]{Merryn Tawhai}

     \affil[*]{Department of Mathematics, Faculty of Science, University of Auckland, Auckland, New Zealand}
     \affil[$\dagger$]{Auckland Bioengineering Institute, University of Auckland, Auckland, New Zealand}
 
     \date{}

     \maketitle

     \begin{abstract}
             The long-term behaviour of complex systems is typically captured by the analysis of non-linear models and their statistical properties.
             Recent works by ecologists and climate scientists have highlighted that these asymptotic equilibria may, from time to time, be subject to extreme disruption due to loss of stability.
             In the past decade mathematicians have sought to identify and quantify this emergent phenomena using the language of dynamical systems and bifurcation theory.
             In the first of this two parts talk we will discover how catastrophic events that occur in nature and society can be described in the developing framework of tipping points under restrictive assumptions.
             We will follow this in part $2$ by introducing a novel area of research that deals with the detection of signs of early warning that precede such catastrophes.
     \end{abstract}
\end{document}
